\documentclass{article}
\usepackage[utf8]{inputenc}
\usepackage{geometry}

% Set margins to zero
\geometry{
  a4paper,
  left=1cm,
  right=1cm,
  top=1cm,
  bottom=1cm,
}
\begin{document}

\section{Vorlesung}
\subsection{IT-System}
\subsubsection{IT-System}
technisches System mit der Fähigkeit zur Speicherung und Verarbeitung von Informationen
\subsubsection{Information}
wird durch Daten repräsentiert und ergibt sich durch eine festgelegte Interpretation der Daten
\subsubsection{Objekte}
\begin{itemize}
    \item \textbf{passive} Objekte (z.B. Dateien): Fähigkeit zur \textbf{Speicherung} von Information
    \item \textbf{aktive} Objekte (z.B. Prozesse): Fähigkeit zur \textbf{Speicherung und Verarbeitung} von Informationen
    \item \textbf{Assets}: Informationen und Objekte, die repräsentiert, sind die schützenswerten Güter (asset) eines Systems 
\end{itemize}
\subsubsection{Subjekte}
\textbf{Benutzer oder aktive Objekte}, die im Auftrag von Benutzern aktiv sind (z.B. Prozesse, Server, Prozeduren)
\subsubsection{Zugriffe}
Interaktionen zwischen einem Subjekt und einem Objekt durch die Informationsfluss auftritt
\begin{itemize}
    \item Zugriff auf Datenobjekt ist gleichzeitig Zugriff auf die dadurch repräsentierte Information
\end{itemize}
\subsection{Sicherheit}
\subsubsection{Funktionssicherheit (engl. safety)}
\begin{itemize}
    \item Ist-Funktionalität == Soll-Funktionalität
    \item Das System funktioniert unter allen (normalen) Betriebsbedingungen
    \item z.B. technische Fehlverhalten des Systems durch Programmierfehler $\Rightarrow$ Programmvalidierung oder -verifikation können es lösen
\end{itemize}
\subsubsection{Informationssicherheit (security)}
    Informationssicherheit ist gegeben, wenn "ein funktionssicheres System nur solche Systemzustände annimmt, die zu keiner \textbf{unautorisierten Informationsverändeurng oder -gewinnung} führen"
\subsubsection{Datensicherheit}
\begin{itemize}
    \item Datensicherheit ist gegeben, wenn "ein funktionssicheres System nur solche Systemzustände annimmt, die zu keinem \textbf{unatuorisierten Zugriff} auf Systemressourcen und insbesondere auf Daten führen"
    \item Umfasst Datensicherung (backup): "Schutz vor Datenverlust durch Erstellung von Sicherungskopien"
\end{itemize}
\subsubsection{Privatheit, Datenschutz}
natürliche Person kontrolliert Erhebung und Verarbeitung ihrer persönlichen Daten
\begin{itemize}
    \item Informationelle Selbstbestimmung
\end{itemize}
\subsubsection{Verlässlichkeit}
Funktionssicherheit + Funktion wird zuverlässig erbracht 
\subsection{Schutzziele}
Welche Funktionen können wir implementieren, um ein informationssicheres bzw. datensicheres System zu haben?
\begin{itemize}
    \item Identifikation und Verifikation der Identität der zugreifenden Subjekte
    \item Zugriffseinschränkung und -kontrolle
    \item Zuordung von Aktionen und Zugriffen zu zugreifenden Subjekten
\end{itemize}
\subsection{Authentizität}
\subsubsection{Authentizität}
\begin{itemize}
    \item "Echtheit und Glaubwürdigkeit des Objekts bzw. Subjekts, die anhand einer \textbf{eindeutigen Identität und charakteristischen Eigenschaft} überprüfbar ist"
\end{itemize}
\subsubsection{Authentifikationen}
\begin{itemize}
    \item \textbf{Nachweis}, dass eine behauptete Identität eines Objekts bzw Subjekts mit dessen charakterisierenden Eigenschaften übereinstimmt
    \item z.B Benutzererkennungen, Benutzernamen mit Passwörtern, biometrische Merkmale als Eigenschaften
\end{itemize}
\subsection{Datenintegrität}
\subsubsection{Datenintegrität}
ist "[...] ist gewährleistet, wenn es Subjekten nicht möglich ist, die zu schützenden Daten \textbf{unautorisiert und unbemerkt} zu manipulieren"
\begin{itemize}
    \item Unautorisiert $\Rightarrow$ \textbf{Rechtefestlegung} z.B. Lese- oder Schreiberechtigungen für Dateien
    \item Unbemerkt $\Rightarrow$ \textbf{Manipulationserkennung}. Manipulationen sind nicht vermeidbar, aber müssen erkannt werden (z.B. Hashfunktionen)
\end{itemize}
\subsection{Vertraulichkeit}
"[ist] gewährleistet, wenn [...] \textbf{keine unautorisierte Informationsgewinnung} [möglich ist]"
\begin{itemize}
    \item Unautorisiert $\Rightarrow$ \textbf{Berichtigungen, Zugriffsrechte, und Kontrolle}
    \item Unautorisiert $\Rightarrow$ \textbf{Verschlüsselung}
\end{itemize}
\subsection{Verfügbarkeit}
\begin{itemize}
    \item .[ist] gewährleistet, wenn \textbf{authentifizierte und autorisierte} Subjekte in der Wahrnehmung ihrer Berechtigungen \textbf{nicht} unautorisiert \textbf{beeinträchtigt} werden können"
    \item d.h. "Normale" Nutzer verfügen unter normalen Bedingungen über die Ressourcen des Systems
    \item z.B. Einführung von Quoten für CPU-Zeit oder Speicher 
\end{itemize}
\subsection{Verbindlichkeit}
\subsubsection{Verbindlichkeit bzw. Zuordbarkeit}
\begin{itemize}
    \item "[ist] gewährleistet, wenn es nicht möglich ist, dass ein Subjekt im Nachhinein die \textbf{Durchführung einer solchen Aktion abstreiten} kann"
    \item D.h. Die Aktionen eines Nutzers können zu seiner Person zugeordnet werden
    \item z.B. \textbf{Digitale Signaturen}
\end{itemize}
\subsection{Inhärente Zielkonflikte}
Um die \textbf{Vertraulichkeit} von Informationen zu schützen kann die Löschung der Information (Selbstzerstörung) angebracht sein
$\Rightarrow$ Verlust der \textbf{Verfügbarkeit} \\
Um die \textbf{Verfügbarkeit} von Informationen zu schützen, können Backup-Kopien von vertraulichen Informationen (z.B. Passwörter PINS) angebracht sein
$\Rightarrow$ Erhöhtes Risiko des Verlusts der \textbf{Vertraulichkeit}
\end{document}